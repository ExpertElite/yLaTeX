\printMiniToc

\section{Diffusion}
Si \nomJeu\ vient un jour à être distribué, ce sera gratuitement et librement, bien sûr! Chacun pourra télécharger le jeu et l'essayer sans payer. Mais plus encore, chacun sera libre de télécharger et modifier son contenu puis de le redistribuer (sous la même licence). Il pourrait même être envisageable de rendre le jeu public avant qu'il ne soit achevé et demander de l'aide à la communauté pour le terminer mais cela implique beaucoup de coordination et une certaine perte de contrôle sur la version \textit{officielle} du jeu.


\section{Réaliser un jeu vidéo libre?}
Pour la réalisation de ce jeu, je m'étais fixé comme condition de n'utiliser que des logiciels gratuits. Atteindre cet objectif fut relativement simple. Il existe, de par le Net, une multitude de programmes gratuits et c'est sans difficulté que j'ai trouvé les outils qui m'étaient nécessaires.

Pour ce qui est du deuxième objectif, à savoir utiliser le plus possible de logiciels libres, je fus agréablement surpris de constater qu'ils sont également nombreux et divers. En effet, il existe, pour la plupart des tâches que nécessite la création d'un jeu, des outils open source. Un seul programme a fait exception à la règle pour \nomJeu: Sketchup. J'avais besoin de ce dernier pour convertir les modèles téléchargés d'Internet vers des formats plus aisément éditables. Impossible de s'en passer donc et impossible de trouver une alternative. Mais ce n'est qu'une goutte au milieu de l'océan.

À mon avis, les objectifs fixés au début de ce document concernant l'aspect technique sont atteints. C'est pour moi la preuve qu'Internet pourrait apporter une révolution dans notre mode de fonctionner: un monde où \textit{la communauté} revêtirait une importance toute particulière, où l'information serait libre et gratuite et où les outils seraient disponibles pour tous ceux qui en auraient besoin.


\section{Réaliser un jeu vidéo seul?}
La création d'un jeu vidéo se déroule principalement en deux étapes que la structure de ce TM  reflète: la préparation sur le papier du design, du scénario, des personnages, etc. puis la réalisation technique avec ses contraintes et problèmes. Commencer un jeu sans idée claire de la direction qu'il va prendre est la meilleure façon de faire beaucoup de travail inutile et de ne jamais aboutir à rien.

Mon scénario n'est bien sûr pas terminé. Cependant, je pense qu'il est suffisamment détaillé pour permettre la création des premiers niveaux. Il pose en tout cas les bases du gameplay, des personnages et dans une moindre mesure l'apparence d'Éluria. Il m'aura fallu pas mal de temps pour en arriver là: le scénario n'a vraiment pris forme que durant l'été. Mais cette réflexion m'a été très utile pour arriver à un tout cohérent et conforme aux buts que je voulais donner à mon jeu. Par exemple, éliminer l'aspect violent peut se révéler être très difficile.

L'expérience fut moins concluante du point de vue de la réalisation technique. S'il était évident que je ne pouvais finir un jeu complet en moins d'une année, j'aurais aimé mener ce projet plus loin, dans le temps imparti. Certaines tâches qui me semblaient aisées de prime abord m'ont parfois demandé des heures de recherches et certains \anglicisme{bugs} m'ont coûté un temps important et précieux. Mais plus encore, je remarque que certaines tâches très chronophages de ce travail, tel l'apprentissage des règles et du langage de Godot n'apparaissent pas dans le résultat final. C'est, d'une certaine façon, la partie immergée de l'iceberg, et ces nécessités représentent la part ingrate de ce travail car rares sont les personnes qui mesurent l'engagement nécessaire pour accomplir ces tâches.

Je rends donc un scénario conforme à mes espérances mais une esquisse du jeu moins avancée que ce que j'espérais. Cependant, les bases sont posées. Le scénario est défini dans les grandes lignes et le code de base du jeu est écrit. Mon Travail de Maturité se termine, ce n'est pas le cas du projet \nomJeu. Mon intention est de continuer ce jeu et qui sait, dans quelques années, de le publier si j'arrive un jour à le terminer.


\section{Bilan personnel}
Ce Travail de Maturité fut pour le moins riche en expériences, certaines très positives, d'autres moins. Pour être honnête, il fut pour moi source de stress; j'aurais voulu porter la réalisation du jeu plus loin, ajouter plus de détails, de fonctionnalités. S'engager dans un travail d'une telle ampleur avec la volonté de le mener à bout peut causer des tensions considérables, surtout au vu des difficultés qu'a pu représenter cet objectif.

Évidemment, l'informatique ne serait pas une si belle science sans la part d'énervement qui vient avec. Au cours de cet exercice, combien de fois n'ai-je pas pesté contre les circuits de cette machine tellement complexe qu'est l'ordinateur. Parfois très frustrantes, les contrariétés qui formèrent tout de même la plus grande partie de ce travail furent d'incessantes sources d'irritation.

Il m'a parfois fallu faire de gros efforts pour rester motivé. Apprendre le fonctionnement de Godot (vous trouverez, pour vous faire une idée de sa taille, la documentation du moteur de jeu à l'adresse \url{https://github.com/okamstudio/godot/wiki}) fut probablement un premier obstacle. Mais je suppose que le pire fut la modélisation. Les nombreux problèmes, souvent peu ou mal documentés, rencontrés durant cette étape furent autant de sources de démotivation: bugs de textures, sommets dupliqués, difficultés à trouver des convertisseurs entre Sketchup et Blender, problèmes avec les formats de fichier à l'exportation de Blender vers Godot, objets trop lourds faisant surchauffer le moteur de jeu, création et exportation des animations des personnages; la liste est encore longue.

Mais il est évident que ce travail m'a aussi beaucoup apporté. J'ai dû persévérer malgré les difficultés et l'énervement. Ce fut aussi l'occasion d'acquérir toutes sortes de connaissances techniques. Dans l'univers de l'informatique et de la création 3D bien sûr, mais aussi dans d'autres domaines plus variés comme l'édition de son et la création de cartes géographiques (et toutes les mathématiques associées à ce dernier thème).

Cet écrit représenta une opportunité pour moi d'exercer une autre de mes passions, la typographie. Réalisé avec \XeLaTeX, un langage de description très puissant, la mise en page de ce document fut une sorte de laboratoire d'essai, l'occasion d'en découvrir (beaucoup) plus sur ce programme fantastique, développé dans les années 80 et pourtant si actuel dans l'univers rapidement changeant de la technologie, qu'est \TeX\ (\XeLaTeX\ étant une sur-couche de \LaTeX\ (développé par Leslie Lamport), lui-même une sur-couche de \TeX\ (développé par Donald Knuth)). J'ai pris beaucoup de plaisir à faire des recherches à ce sujet, choisir des fontes adaptées, créer des tableaux, de même qu'une configuration de page très particulière et des débuts de chapitre un peu fous.

J'ai aussi découvert l'univers du game designer, je suis en quelque sorte passé de l'autre côté du miroir. De joueur, je suis devenu créateur et je n'en apprécie que plus les jeux réussis tant graphiquement que du point de vue du scénario. Ce fut aussi l'occasion de mieux envisager la dichotomie du processus de création d'un jeu: l'aspect purement créatif et ses idées folles souvent peu réalisables, et le point de vue technique de l'informaticien, qui tous deux furent mes réalités à différents moments.

La création d'un univers, outre son aspect fortement divertissant, m'a permis d'en apprendre plus sur mes propres processus créatifs. Mais peut-être plus important encore, ce premier aperçu de la création d'un univers, cet exercice d'imagination m'a donné la chance d'apprendre à me connaître. Les sujets importants pour moi, les personnages que j'ai créés, l'histoire telle que je l'ai conçue; tous ces éléments, finalement, sont des reflets de mes valeurs et me permettent de me découvrir.
