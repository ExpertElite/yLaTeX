\documentclass[a4paper, 11pt, twoside]{article}
\usepackage[utf8]{inputenc}
\usepackage[T1]{fontenc}
\usepackage[francais]{babel}

\usepackage[usenames, dvipsnames, svgnames, table]{xcolor} %nouvelles couleurs, couleurs personnalis�e
\usepackage{graphicx} %gestion des images
\usepackage{csquotes} % pour les citations (guillemets adapt�s � la langue, etc.), commande: \enquote{text}
\usepackage{url} %url
\usepackage{blindtext} %pour g�n�rer du texte
\usepackage{setspace} %gestion des interlignes
\usepackage{fancyhdr} %header personnalis�
\usepackage{multicol} %g�re les colonnes multiples (plus de deux)
\usepackage{wrapfig} %permet d'avoir des images avec du texte qui s'enroule autour
\usepackage{lettrine} %pour les capitales initiales
\usepackage{mathptmx} %police times
\usepackage{rotating} %pour pouvoir faire tourner du texte (90�, 180�, etc)
\usepackage[normalem]{ulem} %pour la gestion de tous les soulign�s
\usepackage[large]{cwpuzzle} %pour les sudokus
\usepackage{pas-crosswords} %pour les mots-crois�s
\usepackage{tcolorbox} %Pour les boites avec des angles arrondis et des couleurs
\usepackage{pgfornament} %pour pouvoir ajouter des ornements
\usepackage{cwpuzzle}

																		%%%%%%%% Mes commandes/instructions			
		%%%%%%%% Mes couleurs
\definecolor{DarkGray}{gray}{0.3}


		%%%%%%%% Package tcolorbox
\makeatletter
%Boite pour les rubriques a bords longs (pas utilis�)
\newtcbox{\encadrerond}{on line,arc=3pt,colback=black,colframe=black,boxrule=0pt,boxsep=0pt,left=6pt,right=6pt,top=4pt,bottom=4pt}
\makeatother


		%%%%%%%% Mes commandes
		
%Titre, sous-titres, rubrique, stamp, auteur et lieu
		%formater un titre
		\newcommand{\titre}[1]{\vspace*{-3.5ex}\textbf{\LARGE{{\fontfamily{phv}\selectfont\begin{spacing}{0.9} #1\end{spacing}}}}}

		%formater l'auteur
		\newcommand{\auteur}[1]{\vspace*{-3mm}\begin{small}\begin{bf}#1\end{bf}\end{small}}
		
		%formater l'auteur dans un article � une ou deux colonnes
		\newcommand{\auteurunedeuxcolonne}[1]{\begin{small}\begin{bf}#1\end{bf}\end{small}}

		%formater le lieu
		\newcommand{\lieu}[1]{, \small{#1}\vspace*{-1.5ex}}

		%sous titres dans un article(titre de paragraphes)
		\newcommand{\soustitre}[1]{\raisebox{0.3ex}{$\blacktriangleright$}\hspace{1mm}\textbf{\large{#1}}\\}

		%th�me/mot-cl� d'un article
		\newcommand{\stamp}[1]{\hfill\colorbox{black}{\color{White}\textbf{\normalsize{#1}}}\vspace{-1mm}}
		
		%th�me/mot-cl� d'un article avec des bords ronds
		\newcommand{\stamprond}[1]{\hfill\encadrerond{\textbf{{\color{white}{\normalsize #1}}}}}
		
		%rubrique (Edito, L'essentiel, Jeux)
		\newcommand{\rubrique}[1]{{\fontfamily{phv}\selectfont\LARGE{{\color{DarkGray}\vspace*{-0.2ex}#1}}}}
		
		%Carr� de fin d'article
		\newcommand{\finarticle}{{\color{DarkGray}\hspace{1em}$\blacksquare$}}
		
		%Indication que l'article continue sur la page suivante
		\newcommand{\suite}{{\color{DarkGray}\hspace{1em}\textbf{ >\hspace{.3ex}>\hspace{.3ex}>}}}

%Commande et environnement article
		%environement pour des articles avec colonnes balanc�es
				%Utilisation: \begin{article}[nombre de colonnes]{Titre}{Auteur}{Lieu} Texte de l'article \end{article}
		\newenvironment{article}[4][3]
			{\titre{#2}\auteur{#3}\lieu{#4}\begin{multicols}{#1}}{\end{multicols}}

		%environement pour cr�er des articles avec colonnes non balanc�es
				%Utilisation: \begin{article*}[nombre de colonnes]{Titre}{Auteur}{Lieu} Texte de l'article \end{article*}
		\newenvironment{article*}[4][3]
			{\titre{#2}\auteur{#3}\lieu{#4}\begin{multicols*}{#1}}{\end{multicols*}}
		
		%environnement pour des articles � deux colonnes avec une meilleure gestion de l'espace entre le titre et l'auteur		
		\newenvironment{articlebis}[4][3]
			{\titre{#2}\auteurunedeuxcolonne{#3}\lieu{#4}\begin{multicols}{#1}}{\end{multicols}}	
		
		%environnement pour des articles � deux colonnes avec une meilleure gestion de l'espace entre le titre et l'auteur		
		\newenvironment{articlebis*}[4][3]
			{\titre{#2}\auteurunedeuxcolonne{#3}\lieu{#4}\begin{multicols*}{#1}}{\end{multicols*}}	
		
		%cr�er des articles � une seule colonne
				%Utilisation: \articleunecolonne{Titre}{Auteur}{Lieu} Texte de l'article
		\newcommand{\articleunecolonne}[3]
			{\begin{spacing}{1.2}\titre{#1}\end{spacing}\auteurunedeuxcolonne{#2}\lieu{#3\vspace*{2ex}}}
			
		%afficher deux articles, sur deux puis une colonne
			%Utilisation: \doublearticledeuxun{Texte1}{Texte2}
		\newcommand{\doublearticledeuxun}[2]
			{\begin{minipage}[t]{0.6466\linewidth}\setlength{\parskip}{0.8ex}#1\end{minipage}\hspace{0.02\linewidth}\vrule\hspace{0.02\linewidth}\begin{minipage}[t]{0.3133\linewidth}\setlength{\parskip}{0.8ex}#2\end{minipage}\vspace{4mm}}

		%afficher deux articles,sur une puis deux colonnes
			%Utilisation: \doublearticleundeux{Texte1}{Texte2}
		\newcommand{\doublearticleundeux}[2]
			{\noindent\begin{minipage}[t]{0.3133\linewidth}\setlength{\parskip}{0.8ex}#1\end{minipage}\hspace{0.02\linewidth}\vrule\hspace{0.02\linewidth}\begin{minipage}[t]{0.6466\linewidth}\setlength{\parskip}{0.8ex}#2\end{minipage}\vspace{4mm}}

%Initial capitals et lignes

		%initials capitals et small caps
		\newcommand{\icsc}[2]{\lettrine{#1}{#2}} 

		%lignes horizontales de s�paration entre les articles
		\newcommand{\ligne}{\hrule\vspace{2.5ex}}

%Encadr�s et pub
		%encadr�s de plus d'une colonne
			% Utilisation: \encadrecolonnes{noDeColonnes}{Titre de l'encadre}{texte de l'encadre (supporte les commandes)}
		\newcommand{\encadrecolonnes}[4][0.968]
			{\fbox{\parbox{#1\linewidth}{\setlength{\parskip}{.8ex}\vspace*{1ex}\titre{#3}\vspace{-1ex}\begin{multicols}{#2}#4\end{multicols}}}}

		%encadr�s d'une seule colonne
			%Utilisation: \encadre{Titre de l'encadre}{Texte de l'encadre (supporte les commandes)}
		\newcommand{\encadre}[3][0.90]
			{\vspace{2mm}\noindent\fbox{\parbox{#1\linewidth}{\setlength{\parskip}{0.8ex}\vspace*{1ex}\begin{spacing}{1.30}\titre{#2}\end{spacing}\vspace*{1ex}#3}}}
	
		%petits �critaux de pub
		\newcommand{\petitepub}[3]{\vspace{3mm}{\color{Gray}\rule{\linewidth}{1.5mm}}\\[1mm]\begin{minipage}{1.2cm}\includegraphics[width=1.2cm, height=1.2cm]{#1}\end{minipage}\hspace{2mm}\begin{minipage}{4.5cm}\textbf{#2}#3\end{minipage}\nopagebreak\vspace{1mm}\hrule}

%Mises en exergue
		%mises en exergue au d�but d'un article
		\newcommand{\exerguedebutarticle}[2][]{{\fontfamily{phv}\selectfont\textbf{{\color{DarkGray}\textsc{#1}}}}\vspace*{1mm}\hrule\vspace*{-1mm}{\fontfamily{phv}\selectfont\textbf{{\color{Gray}\raisebox{.4mm}{> }}#2}}\vspace*{1mm}\hrule}
		
		%mise en exergue au d�but d'un article � deux colonnes
		\newcommand{\exerguedebutarticledeuxcolonnes}[2][]{{\fontfamily{phv}\selectfont\textbf{{\color{DarkGray}\textsc{#1}}}}\vspace*{1mm}\hrule\vspace*{1mm}{\fontfamily{phv}\selectfont\textbf{{\color{Gray}>} #2}}\vspace*{1mm}\hrule}

		%exergue de d�but d'article � une colonne
		\newcommand{\exerguedebutarticleunecolonne}[2][]{\vspace*{2.5ex}\\{\fontfamily{phv}\selectfont\textbf{{\color{DarkGray}\textsc{#1}}}}\vspace*{1mm}\hrule\vspace*{-2ex}\begin{flushleft}{\fontfamily{phv}\selectfont\textbf{{\color{Gray}> }#2}}\vspace*{-1.6ex}\end{flushleft}\hrule\vspace*{1ex}}

		%nouveau paragraphe dans l'environnement exerguedebutarticle
		\newcommand{\newpar}{\\\vspace{0.8ex}{\color{Gray}{\textbf\raisebox{.4mm}{> }}}}

		%mise en exergue au milieu des articles
		\newcommand{\exerguemilieuarticle}[1]{\vspace*{2ex}\hrule\vspace*{-.5ex}\textsc{{\raggedright{}{\fontfamily{phv}\selectfont\large{#1}}}}\vspace*{1.5ex}\hrule\vspace*{1ex}}

%R�sum�s d'articles
		%r�sum� d'articles � mettre sur la page de garde pour attirer les lecteurs
		\newenvironment{essentiel}[1][2]{\rubrique{L'essentiel...}\vspace*{-2ex}\begin{multicols*}{#1}}{\end{multicols*}}

		%r�sum� de chaque article dans l'anvironement essentiel
		\newcommand{\resume}[4]{{\fontfamily{phv}\selectfont\small{\color{DarkGray} #1}}\vspace*{.5ex}\hrule\vspace*{-2mm}{\fontfamily{phv}\selectfont{\large\textbf{#2}}}\\#3\\{\small\textbf{Page #4}}\vspace*{1.5ex}}

%Jeux
		%d�finitions du mot crois�
		\newcommand{\maDef}[2]{\printDef{#1}{\hspace{-2ex}\textbf{.} #2}}

%Soulignement et url
		%url
		\newcommand{\myurl}[1]{{\color{blue}\uline{\url{#1}}}}
		
		%souligner avec des points
		\newcommand{\udensdot}[1]{\tikz[baseline=(todotted.base)]{\node[inner sep=1pt,outer sep=0pt] (todotted) {#1};\draw[densely dotted] (todotted.south west) -- (todotted.south east);}}

		%souligner avec des traits till�s
		\newcommand{\udensdash}[1]{\tikz[baseline=(todotted.base)]{\node[inner sep=1pt,outer sep=0pt] (todotted) {#1};\draw[densely dashed] (todotted.south west) -- (todotted.south east);}}


		%%%%%%%% Modification des sudokus
\renewcommand{\SudokuLinethickness}{1pt} %Change la largeur de la ligne des sudokus


		%%%%%%%% Modification des encadr�s
\setlength{\fboxsep}{1.5ex} %augmente l'espace entre le texte et les encadr�s


		%%%%%%% Package ulem(soulignage)
\renewcommand{\ULdepth}{2pt} %Soulignage pour les url (la ligne est plus proche du texte que pour un soulignage normal)


		%%%%%%%% Package url
\urlstyle{sf} %tt, rm, sf, same


		%%%%%%%% Package Lettrines
\setlength{\DefaultNindent}{0.5mm}
\setcounter{DefaultLines}{2}
\renewcommand{\DefaultLraise}{0} %Elevation de la lettrine par rapport � la basline
\renewcommand{\DefaultLoversize}{0.3} %facteur de grossissement de la lettrine vers le haut
			%Les deux commandes suivantes sont � utiliser si on veut utiliser une police sp�ciale pour les lettrines
			%\newcommand*\initfamily{\usefont{U}{Konanur}{xl}{n}}
			%\renewcommand{\LettrineFontHook}{\initfamily}


		%%%%%%%% Colonnes
\setlength{\columnsep}{0.5cm} %longueur de l'espacement entre les colonnes


		%%%%%%%% Les longueurs du document
%les longueurs de la feuille
%\setlength{\topmargin}{-.54cm} %marge � ajouter en haut en plus de l'inch deja present
%\setlength{\oddsidemargin}{-.1cm} %distance entre le texte et l'inch de marge du bord des pages impaires
%\setlength{\evensidemargin}{-.1cm} %distance entre le texte et l'inch de marge du bord des pages paires
\setlength{\headsep}{.8cm} %Distance entre le header et le texte
\setlength{\headheight}{13.6pt} %hauteur du header
\setlength{\marginparsep}{0mm}
\setlength{\marginparwidth}{0mm}
%\setlength{\textwidth}{16.56cm}
%\setlength{\textheight}{23.7cm}
\setlength{\footskip}{1.3cm}
\usepackage[
	a4paper,
	%width=15cm, %coh�rent avec les valeurs de marges
	%height=22cm,
	ignoreheadfoot,
	vcentering,
	hmargin=2cm,
	vmargin=3cm,
	footskip=1.5cm
	]{geometry}

% Les longueurs des paragraphes
\setlength{\parskip}{1em} %espace en dessus du paragraphe (entre paragraphes)
\setlength{\parindent}{0pt}  %la longeur de l'alin�a au d�but des paragraphes



		%%%%%%%% Le header
\pagestyle{fancy}
 \lhead{{\bf Le Parchemin}}
	\cfoot{\thepage} %enlever le num�ro de page qui est au milieu du footer


\rhead{janvier 2015}

\fancypagestyle{noNumber}{
	\fancyhf{}
	\rhead{janvier 2015}
	\lhead{{\bf Le Parchemin}}
}

\begin{document}


\newgeometry{ignoreheadfoot,
	vcentering,
	hmargin=2cm,
	vmargin=3cm,
	bottom=2cm	
	}

\thispagestyle{empty}
\vspace*{-3cm}{\centering\includegraphics[width=\textwidth]{../../images/logoParchEpure.png}}

\vspace*{-.8cm}\rule{0.87\textwidth}{1pt}\hspace{3pt}{\fontfamily{phv}\selectfont janvier 2015}

\begin{center}
\vspace*{-.5cm}
\includegraphics[width=\textwidth]{images/couverture2.jpg}
\end{center}

\begin{spacing}{.9}
\vspace*{-1.1cm}
\begin{article}{Edito}
{Alicia Frésard}{salle de rédaction}
\begin{spacing}{.96}
\icsc{C}{e début d'année} 2015 aura été riche en émotions... A peine rentrés de vacances, aussi frais que des oisillons tombés du nid, nos pauvres esprits ont tous été secoués par un véritable déferlement d’événements. 

L'attente insoutenable des résultats aux examens de décembre (particulièrement difficiles pour les premières années, dont c'était, rappelons-le, la première session. Aller, plus que 7 et c'est fini) : déceptions, joies, rires, larmes, reniflements, neutralité, beuverie. Tant de faits que l'on ne voudrait jamais avoir eu besoin de vivre.\end{spacing}

Et puis la dernière volée de collégiens, nous, petits quatrièmes, avons couru la fin d'un marathon (presque) silencieux : la phénoménale course aux soutenances de travaux de maturité. Entre retards, reports et mauvaises attributions de classe, celles-ci ont été la cause de quelques frissons au sein de notre communauté. Mais au final, c'est avec un très grand, un énorme soupir de soulagement que tout s'est enfin terminé. Le sort n'est plus entre nos mains, il faut maintenant attendre.

Et puis finalement, cette première semaine de cours a également été le témoin d'un événement qu'il nous faut mentionner, en notre qualité de journal et de symbole de l'expression : l’attentat du 7 janvier, dont ont été victimes 12 personnes. Chaque mois, nous sommes nous-mêmes confrontés aux limites de notre liberté d'expression. C'est un acte terrible, qui a cependant été surmédiatisé, élevant cet événement au rang de plus grosse controverse de ce début d'année. Néanmoins, le Parchemin rend hommage à ces personnes, qui ont choisi pour arme un crayon, en lui consacrant sa couverture, plutôt que des mots.

Et bien sûr, le Parchemin vous envoie pleins de bisous et vous souhaite une jolie année 2015 !
\end{article}
\end{spacing}

\restoregeometry
\newpage


\vspace*{-1.7cm}
\begin{article}{Steve Hunziker, le bibliothécaire ! \stamp{Interview}}
{Lino \& Garance}{salle ultrasecrète de la bibliothèque}

\includegraphics[width=\linewidth]{images/interview2.jpg}

\icsc{\textbf{T}}{\textbf{out d'abord}}\textbf{, qui êtes-vous ?}\\
Je m'appelle Steve Hunziker, j'ai 45 ans, je suis bibliothécaire documentaliste et je travaille ici depuis trois mois.
Je suis «the average guy», j'essaie de m'intéresser à tout, je suis assez ouvert, et j'essaie de travailler, d'avoir une famille, des amis, simplement. 
J'ai aussi été rédacteur en chef d'un journal au collège. C'était à Neuchâtel, on a publié quelques numéros. Dans le deuxième numéro, on a eu de grands problèmes, car on avait reçu une commandante de l'armée kurde et on a voulu faire un article sur ce sujet très intéressant, mais le directeur nous en a empêché parce que ça faisait scandale, donc on a été censurés. Il a fallu peser le pour et le contre, et on s'est peut-être laissé un peu impressionner par la direction, ce qu'on n'aurait pas dû, mais on était un peu jeunes... 

\textbf{À votre avis, Internet va-t-il tuer les bibliothécaires ?}\\
Le métier de bibliothécaire comme il est représenté dans certains livres, films ou séries, c'est-à-dire - (\textit{interruption par Mme Purro qui vient chercher ses affaires !}) - le vieux ou la vieille bibliothécaire acariâtre, qui n'aime pas le contact avec les autres et qui aime conserver ses livres, c'est en train de disparaître, si ça n'a pas complètement disparu. Les bibliothèques sont en train de changer. Dans les centres de documentation, il y a de plus en plus de sources numériques, mais les sources papier resteront toujours là. Les bibliothécaires offriront de nouveaux services, comme aider à trouver des informations, et le centre de documentation restera toujours un lieu de transmission du savoir. L'objet livre restera lui aussi, car il y aura toujours des bibliophiles. Ou je ne sais pas, peut-être que dans deux cents ans, le livre aura disparu... 

%\exerguemilieuarticle{Quel personnage emblématique de roman seriez-vous ?\\-- Sherlock Holmes}

\textbf{Quel est votre auteur favori ?}\\
J'aime beaucoup la science-fiction, et je pense que le livre qui m'a le plus marqué, c'est Dune de Frank Herbert. C'est une saga de 5 volumes, qui décrit un univers, des planètes, des langues, des systèmes politiques, des religions, et c'est génial, c'est écrit magnifiquement. J'aime aussi lire du Colette, ou bien un auteur que j'adore vraiment, dont j'ai tout acheté : Nathalie Sarraute. Elle s'est penchée sur ce qu'il se passe à travers les phrases quand les gens discutent. Son livre le plus connu est \textit{Tropismes}, mais vous avez autre chose au programme, j'imagine ! (\textit{rires}) C'est fin, c'est intelligent... c'est pas chiant ! (\textit{re-rires})

\textbf{Quel personnage emblématique de roman seriez-vous ?}\\
Sherlock Holmes (\textit{admiration des journalistes : «Excellent choix !»}). C'est à la fois un scientifique, un écrivain, un enquêteur, un homme impossible, qui n'arrête pas de sauter dans tous les coins, un violoniste. Oui, Sherlock Holmes, définitivement.

\textbf{Vous ne vous ennuyez pas trop ?}\\
Alors non, moi je ne m'ennuie jamais. J'ai la chance de faire un métier qui est aussi une sorte de passion. Lire, c'est passionnant, mais je ne lis pas tout le temps : je fais aussi des listings, je gère le budget. J'ai travaillé dans plein d'endroits différents, dans la tour d'un château, en librairie en tant que spécialiste du rayon ésotérisme, dans le secteur de l'horlogerie, des outils médicaux ou avec des chercheurs très pointus, et je crois qu'on peut s'intéresser à tout.

\soustitre{Mythe ou réalité}[.7\baselineskip]
\textbf{La bibliothèque a une odeur unique au monde.}\\
Non... J'ai travaillé dans des endroits où il y avait vraiment une odeur particulière. Ici non, sauf peut-être dans la salle du fond (la salle Pierre Bruchez, NDLR), c'est une histoire de moquette, je crois, qui aurait été nettoyée avec un produit bizarre et il y a une odeur très étrange qu'on n'arrive pas à faire disparaître. 

\textbf{La consommation de papier d'imprimante tue vraiment un bébé phoque par jour.}\\
Cette affiche est ridicule et je pense qu'elle n'atteint pas son objectif. Elle fait plus rire que compatir. On va la changer. Je ne sais pas comment faire comprendre aux élèves qu'il faut faire un effort pour imprimer moins. Souvent, il y a un tel gaspillage, les gens impriment vingt pages alors qu'ils ont besoin de deux, alors qu'ils pourraient juste régler les paramètres d'impression. Heureusement, il y a un groupe dans l'école auquel je vais faire appel. Peut-être qu'ils auront une bonne idée, une solution. Donc c'est une réalité. Un bébé phoque, ou un arbre, meurt chaque jour en poussant des hurlements !

\textbf{Fumer la moquette de la bibliothèque permettrait d'augmenter sa moyenne de français.}\\
Je pourrais répondre plein de choses, parce que je vois beaucoup de choses... Mythe, mythe, mythe, mythe ! (\textit{rires}) J'ai fumé, j'ai jamais eu de bons résultats en fumant, ça ne m'a jamais permis de bien bosser. Après, à chacun de savoir s'il a envie de fumer la moquette. Ne mettez pas le feu quand même !

\textit{Merci à notre bibliothécaire d’avoir pris le temps de répondre à nos questions!}
\end{article}


\newpage




\vspace*{-2cm}\begin{article}{Labor Maxime Inutilis Maturitatis\stamp{TM}}
{Mr et Mrs Spartacus}{depuis la Tour d’Ivoire}
\exerguedebutarticle[Documentaire animalier]{En expédition dans les coins reculés de la planète Destaël, des reporters en découvrent plus sur ce mystérieux être qu'est le TM...}

\vspace*{1mm}
\icsc{« J}{e m’appelle} Analyse de marché des exports internationaux de la production proto-agricole de patate douce en Éthiopie entre 1963 et 1965 / Deux semaines dans la peau d’un moine copiste cistercien / La cigarette: un danger ?/ Je suis né le brumeux mardi 11 novembre 2014 à 7h30 / 11h30 / 12h20 (accouchement difficile). Je pèse 46g / 15g / 120kg (réalisation artistique absconse). Ma maman / mon papa / mes parents nage(nt) dans le bonheur /  est/sont content(s) d’avoir enfin fini / est/sont au bord de la dépression.» (Biffez les mentions inutiles)

Voilà ce sur quoi vous risquez de tomber si vous vous aventurez dans les antres solitaires de la garderie des TM, aussi connue sous le nom de Centre de documentation. 

Ce mois-ci, dans notre habituelle rubrique animalière, nous allons étudier un énergumène typiquement genevois : le TM. Il s’agit là bien sûr d’une barbare abréviation, comme vous le savez tous, désignant le très célèbre \textit{Labor Maxime Inutilis Maturitatis}. Mais pour des raisons évidentes, nous allons nous contenter d’user de cette abréviation, aussi réductrice soit-elle. Reprenons. Le temps de gestation du TM commun est très variable; cela peut aller de celui d’une musaraigne (deux semaines) à celui d’un lamantin d’eau douce peu profonde (une année et demi). Nous avons tout de même observé un temps moyen d’environ dix mois. Du fait de cette disparité, nous nous sommes rendu compte de l'hétérogénéité des spécimens; certains sont précoces, d’autres ont eu besoin de temps pour perfectionner le premier jet. A l’époque du mariage pour tous, on note également qu’un TM ne naît pas uniquement des fruits du laborieux ouvrage d’un homme et d’une femme. Il s’agit certes d’une possibilité, mais la majorité préfère effectuer cette épopée en solitaire. Deux femmes ou deux hommes peuvent également s’associer pour créer un joli petit nourrisson. Même si les temps de gestations peuvent aller du simple au beaucoup, la saison des accouchements est très brève: elle ne dure qu’une seule matinée. La fatidique matinée du 11 novembre.

Nous aurions pu continuer nos passionnantes considérations scientifiques des pages durant encore, mais tout à coup soudain, un TM sauvage apparut et nous renseigna sur les étranges coutumes de ces peuplades primitives, que nous classons sous la charmante appellation latinisante évoquée auparavant (cf. titre). Notre cher cuistre, quelque peu brute de décoffrage, nous apprit que chaque parturient était aidé tout au long de son travail par un \textit{Hemmâ}, bizarrement qualifié de \enquote{Maître}. Mais, me direz-vous – et vous aurez raison! –, n’est-il pas choquant qu’un simple employé du corps médical se voit octroyer l’éminent qualificatif de \enquote{Maître}? Si! Et ce n’est pas tout! Figurez-vous que celui qui vous a aidé, vous a épaulé, vous a soutenu (ou pas, et dans ce cas vous êtes un grand zozo bien majeur) est également celui qui vous juge.
Les pratiques farfelues de ce peuple s’arrêtent-elles là? Ne sommes-nous pas déjà arrivés au-delà du septième cercle des Enfers? Eh bien non! Quelques mois après l’accouchement, un rite religieux saugrenu, que l’on pourrait rattacher dans nos contrées au baptême – ou à la circoncision, c’est selon –, est organisé en comité restreint dans une salle obscure. Le Hemmâ, accompagné d’un de ses sbires, porte d’insidieuses escarmouches au malheureux quidam qui n’a pour souvenir de son TM que de douloureuses contractions. On croirait voir Minos, Eaque et Rhadamanthe en train d’effectuer la pesée. Il s’agit d’un rite initiatique censé préparer à l’obtention du précieux sésame qu’est la Maturité, paraît-il…

Prenez garde, lecteurs, une fois déterminés à devenir parents d’un beau petit TM, il vous faudra assumer votre choix, tant de partenaire que de problématique. L’avortement est malheureusement impossible sur ces terres incertaines. Sortez couverts et filtrez votre vin!
\end{article}

\ligne




\vspace*{-5mm}
\begin{article}
{Tu veux ma photo?}
{Matthis E. Pasche}{derrière l'objectif}

\icsc{« M}{angez-moi! »} C'était le thème, l'an dernier, du printemps carougeois, qui fêtait son cinquantième anniversaire. On trouvait en effet à Carouge, par dizaines, des  activités et expositions tournant autour de notre sujet préféré (ou presque...) : la nourriture ! Des balades gastronomiques, des lectures apéritives, un concours de courts-métrages dont le thème était \columnbreak le miam-miam... Bref, de quoi exciter nos papilles ! Des petits génies avaient même pensé à décorer les pavés de vingt-cinq-mille bougies (non, vous ne pouviez pas les manger).

Or, cette année, Carouge vous propose pour sa fête, dont on ne connaît pas encore le thème, de vous faire entrer dans un photomaton, qui sera installé dans notre cher collège le 30 janvier (et à d'autres endroits de Carouge les jours précédents), d'appuyer sur le bouton : flash... Et voilà, c'est tout, la Ville s'occupera du reste. Un magnifique chef-d'œuvre vous sera sans doute présenté au printemps. Alors surtout, n'hésitez pas à entrer dans la boîte : à de Staël, on est beaux, et on mérite d'être pris en photo !
\end{article}

\newpage




\vspace*{-1.8cm}
\begin{article}
{Black Movie}
{Mathilde}{confortablement assise devant un énième film}
\exerguedebutarticle[Cinéma]{Novateur, décomplexé, aux antipodes des productions commerciales, Black Movie était de retour en toute beauté en janvier. Quelques précisions...}

\icsc{A}{ttentifs} et curieux que vous êtes, vous avez sûrement aperçu au collège ou dans le tram cette affiche d'une vague noire déferlant sur des rayures violettes et oranges... Non? Et bien vous avez raté quelque chose!

Le festival international Black Movie a en effet fait son retour en ce début d'année, avec 10 jours de projections. Vous connaissez? Black Movie propose des films indépendants (pas du Hollywood quoi...) de tous genres et du monde entier. Une réelle occasion de découvrir d'autres horizons et d'autres styles de films que les grosses productions américaines qui ont une très légère tendance à s'imposer.

\exerguemilieuarticle{Cinéphile, participer au Black Movie, c'est fait pour toi !}

Cette année, le festival rendait hommage à un grand réalisateur chinois, auteur entre autres d'un magnifique film d'une toute petite quinzaine d'heures... ça fait des scènes longues, très longues... Mais je vous rassure, les films projetés n'étaient pas tous si longs et il y avait de l'action ! En revanche, on retrouvait dans la plupart un côté noir, sombre, dur et parfois violent, qu'annonçait si bien la vague noire de l'affiche. Des ruelles de prostitution en Chine, jusqu'à un hôpital psychiatrique au Sénégal, en passant par la terreur d'une société argentine approchant de la fin du monde, on ne pouvait pas sortir de ces projections avec une vision rose de la vie…

Si en plus d'avoir vu l'affiche au collège, vous aviez été pris d'un élan de folie et vous vous étiez approchés, vous auriez pu participer au festival, et voir tous ces films gratuitement ! En effet, il existe un jury jeune qui visionne les films afin de les critiquer et d’attribuer un prix au meilleur. Une semaine dispensé d'école à regarder des films ! Pas mal non ? C'est raté pour cette année, mais si vous avez une très bonne mémoire, ce sera peut-être pour celle d'après. Et puis il y a aussi d'autres festivals, auxquels on peut participer ou juste assister: le festival international du film sur les droits humains, pour fin février début mars, Visions du Réel en avril, et bien d'autres encore. Pour toutes les infos, allez faire un tour sur internet !
\end{article}

\ligne



\vspace*{-4mm}
\begin{article}{Critical Mass -- la Vélorution}
{Charlotte, Mathilde et Yves}{île Rousseau}
\exerguedebutarticle[Cyclisme]{Des vélos pleins les rues qui imposent leur rythme au cours d'une soirée endiablée au coeur de Genève : La Critical Mass ! Explications...}

\includegraphics[width=\linewidth]{images/velorution.jpg}

\vspace*{2mm}
\icsc{« I}{l est vendredi soir}, le cortège se met gentiment en mouvement ; la centaine de vélos met le pied à la pédale et s'engage rue du Rhône. La Critical mass est partie ! Les cyclistes se lancent dans les rues de Genève au rythme du reggae, émanant des enceintes bricolées tirées par les manifestants les plus courageux. »

Bon, on vous explique... En gros, la Critical, c'est un mouvement protestataire \textsc{pacifique} international qui veut libérer nos villes de cette épidémie motorisée polluante qui y pullule (Mouhahahaha). Chaque dernier vendredi du mois, simultanément dans plusieurs centaines de villes du monde, une petite foule se retrouve à un endroit donné, équipée d'un moyen de locomotion non-polluant comme un vélo, des rollers, un skate, voire un monocycle pour les originaux. Au programme, balade dans la ville avec quelques arrêts \textit{absolument involontaires} au milieu des ronds-points et carrefours, créant ainsi quelques embouteillages et des réactions différentes du côté des automobilistes à l'arrêt : amusement, énervement mais aussi curiosité et soutient (Bon, c'est vrai, pas mal d'énervement). Le temps d'une soirée, les cyclistes tentent de se réapproprier les rues, ils demandent une meilleure organisation de celles-ci, prône la mobilité douce et dénonce la pollution des voitures en centre-ville. Mais bon, c'est avant tout une super soirée dans une ambiance vraiment sympa.

Si cette petite description vous donne envie, le prochain rendez-vous est prévu pour le vendredi 30 janvier à l'île Rousseau à 18h00, même si le départ est souvent... différé (Y'a du r'tard quoi!). Et c'est chaque dernier vendredi du mois comme ça !

\textsc{Venez nombreux !}


{\petitepub{../../icons/stylos.png}{Vous voulez vous\\exprimer?\\}{Écrivez pour le Parch'!!}}
\end{article}

\newpage






\vspace*{-1.8cm}
\begin{article}
{You are needed !}
{Garance Sallin}{aire Téhesse}
\exerguedebutarticle[Casting]{RTS, casting, Instagram, jeunes. On vous rassure, aucun intrus ne s’est glissé dans ces mots-clés.}

\vspace*{-2mm}
\icsc{L}{es amis}, l’heure est grave. Depuis plusieurs semaines, un événement bouleverse le pays, faisant s’écrouler plus de quatre-vingts ans de tradition, réduisant à néant des lois instaurées depuis longtemps : la RTS est frappée par le démon de midi et recherche de la chair fraîche ! Mais non contente de se tourner vers les jeunes, cette illustre couguar se montre exigeante…

\soustitre{Who ?}
C’est quelqu’un de charismatique qu’il lui faut. Quelqu’un d’énergique, de cultivé et qui peut s’exprimer sur toutes les problématiques liées à la vie des \textit{teenagers}. Quelqu’un de naturel, qui ne se prend pas la tête, et qui n’aura pas peur d’aborder certains sujets, qui seront tant socio-culturels que plus personnels, voire intimes. Autant le dire franchement : le sexe ne devra pas être un tabou !

\soustitre{What ?}
Si tu corresponds à ce profil, que tu as entre 18 et 23 ans et que -- par malheur -- les harmonieuses couleurs du flyer n’ont pas encore attiré ton attention, envoie une vidéo de toi taggée \enquote{\# instacastrts} sur Instagram, où tu expliques pourquoi tu es LA personne dont la RTS a besoin ! 

\soustitre{What else ?}
Pour plus d’informations, rends-toi sur \url{rts.ch/instacast}. 
Pour avoir une idée du ton voulu de la future émission (qui ne se fera que si la perle rare est trouvée !), jette un œil à la chaîne de Laci Green, décomplexée et sans tabou : \url{https://www.youtube.com/user/lacigreen} (intéressé ou pas par le casting, regarde ses vidéos, c’est tout ;) !)

\end{article}

\ligne




\vspace*{-3mm}
\begin{article*}
{Et le thème sera ...\stamp{Impro}}
{Renat Arjantsev}{dans la patinoire}
\exerguedebutarticle[Improvisation théâtrale]{Comment travailler vos abdos régulièrement et gratuitement, résultats garantis !}

\begin{spacing}{1.04}
\icsc{Q}{uand tu sais pas} quoi faire et que dehors c’est déjà l’hiver; c’est le début de la semaine et déjà tu te les pèles ! Donc si tu veux être chaud, va voir un match d’impro ! (Note: lire ces premières lignes avec un ton de rappeur de la cité…) Et là je vois déjà un air d’incompréhension sur votre visage. Mais, au fond, l’improvisation, c’est quoi ?

\soustitre{Origines et concept}
Les matchs d’improvisations théâtrales, car c’est bien de matchs dont il s’agit, ont été inventés par les Québécois (ces gars ont le meilleur accent du monde, non ?!) aux environs de 1977. Afin de casser l’image un peu ennuyeuse que la plupart des gens ont du théâtre, les Québécois ont parodié le hockey sur glace : cela signifie qu’il y a des équipes, donc un championnat avec un classement. Les matchs se déroulent environ toutes les deux semaines (le calendrier des matchs se trouve sur \url{http://www.impro-geneve.ch}) et sont tous gratuits. Le lieu où se déroulent les improvisations s’appelle la patinoire, d’où le lieu de l’article. Chaque équipe à son hymne : des parodies de chansons connues ou des créations originales.

Pour vous faire remarquer l’originalité des improvisateurs, voici quelques noms d’équipes : Les Dindons de la Force, La Villa Tachinni’que Ta Mère ou encore Les Confesseurs. Juste pour info, vous avez des représentants de La Villa Tachinni’que Ta Mère et des Confesseurs dans votre bon vieux collège !

\exerguemilieuarticle{Mais l’impro, quésako ?}

Le principe est simple : un arbitre décide des thèmes (à peu près tout ce qui vous passe par la tête, même les trucs un peu limites … ) et des catégories (par exemple : à la manière de Racine, dramatique, western, ...). Donc chaque impro a un thème, un catégorie, une durée et un nombre de joueurs maximum. Ensuite vient l’improvisation proprement dite et c’est là que vous vous bidonnez un max, puis arrive le moment de compter les points. En fait, c’est le public qui attribue les points ! Donc c’est vous qui allez \enquote{noter} la qualité des improvisations !
Il y a évidemment des fautes comme au hockey, comme par exemple il est interdit d’arriver sur la patinoire uniquement pour tuer tout le monde et se barrer tranquillement. En échange, il est parfaitement possible de jouer un curé qui fait des choses pas vraiment catholiques avec des péripatéticiennes slaves …

Dans un match d’impro, vous allez voir de tout, mais vraiment de tout. La première fois que vous allez voir un match, comme toutes les premières fois, c’est un peu bizarre, mais ça vous plaît et vous avez envie de le refaire. Vous allez finir par adorer et ne pourrez plus vous en passer. Aller, viens nous voir une fois, ça va être bien … (ah et invite tous tes amis, si tu en as, sinon pas grave, viens quand même, on aime tout le monde !)\end{spacing}
\end{article*}

\newpage




\vspace*{-2.1cm}
\begin{article}
{Les Contrepèteries}
{Eric Vladimir}{qui a lu Perceau, horrifié}
\vspace{5mm}
\icsc{C}{hères collégiennes,} chers collégiens, je n’irai pas par quatre chemins: trouvez-vous vos cours de français ennuyeux? Si oui, vous aimerez les contrepèteries! Si non, aussi! Voici venue l’heure de jouer avec notre langue, à la tortiller dans tous les sens, afin de produire des phrases allant de stupides à horribles, en passant par toutes les formes de perversité, point essentiel des contrepèteries. Voici les règles du jeu: dans une phrase à priori banale, inverser des lettres ou des syllabes pour former une nouvelle phrase. En voici un exemple: «\textit{gl}isser dans la \textit{pi}scine» \columnbreak qui devient «\textit{pi}sser dans la \textit{gly}cine».

Maintenant à votre tour de vous entraîner! Attention, on peut inverser des syllabes mais aussi seulement des lettres!

-- Il faut qu’il rende ce vieux bouquin!\\
-- Le grand nombre de monts empêche de les compter.\\
-- Rien ne plaît au matin comme un bon coup de byrrh.\\
-- Le cuisinier secoue les nouilles.\\
-- Il aurait fallu que votre bête fût plus grosse pour que je la prisse.\\
-- Il est tard la mignonne, le voilà dans le pétrin.\\
-- Rien n’est plus gracieux qu’une jeune fille en culotte et en corset.

Et pour terminer, en voici quelques-unes à placer en cours ou dans la vie de tous les jours:\\
-- Le temps abolit les mythes.\\
-- J'aime vachement ton frangin.\\
-- Grand prix de tennis.

Et spécial pour le cours de chimie:\\
-- Oseriez-vous, monsieur, contester nos particules?
\end{article}

\ligne


\vspace*{-5mm}
\begin{article}
{Le TM, tu aimes ?\stamp{TM}}
{La petite inconnue}{perplexe}
\exerguedebutarticle[vie du collège]{Les quatrièmes s’en sont débarrassés, les troisièmes plongent à peine dedans : les TM ne nous lâcheront décidément jamais...}

\icsc{A}{lors que} nos très chers quatrièmes viennent de finir leur soutenance et accèdent enfin à la liberté (ou pas, pour cela il leur faudra attendre six mois), les troisièmes viennent de s’y mettre avec beaucoup d’ambition et de bonne volonté («Ouais, je vais essayer de le finir d’ici juin pour pas avoir à bosser cet été.» «T’inquiète, cette fois-ci je m’organise !»). En effet, l’attribution des MA vient d’être publiée. (NB pour tous les kikoos qui lisent le Parchemin : MA signifie en effet Maître Accompagnant et non pas ta Meilleure Amie besta avec qui tu partages tout). Pour certains, c’est un soulagement et pour d’autres une désillusion, mais, ma foi, je doute que cela change grand-chose puisque c’est avant tout à vous et vous seul de l’écrire, ce fichu travail de maturité ! 

Toutefois, je souhaite finir cet article par un conseil aux prochaines volées de troisième année qui auront la lourde tâche de choisir à leur tour le sujet sur lequel blablater pendant 7'500 mots: choisissez un thème qui vous plaît avant tout, car nombreux sont déjà les élèves à être soûlés par leur travail avant de l’avoir entrepris et parce que, clairement, tout est possible, même les projets typiquement loufoques tels que « la création d’un caquelon à fondue » (Oui, oui, un De Staëlien l’a fait !). Et, pour finir, on vous ne le dira jamais assez, mais réfléchissez-y assez tôt : j’entends d’ailleurs encore mon prof de classe me le dire avec son adorable accent sénégalais il y a deux ans et, pour être honnête, j’ai l’impression que c’était hier !
\end{article}

\vspace*{-.6cm}


\newlength{\retourLigne}
\setlength{\retourLigne}{3mm}


\encadre[.97]{Qui t'a dit ca?}{\vspace*{2mm}
\auteur{Kita Dissa}\lieu{sur le bord de la feuille}

\vspace*{3mm}
\begin{minipage}{.78\textwidth}
1. « Oui avant, quand ils avaient des pantalons-mammouths. »\\
2. « Le 21 vous serez dispensés de cours – j'espère que vous n'êtes pas trop tristes, si jamais j'ai des mouchoirs à disposition. »\\
3. « Il est venu en grandes pompes, mais ça veut pas dire qu'il chausse du 49 ! »\\
4. « L'orthographe, c'est comme le papier toilette : on n'est pas obligé de l'utiliser, mais les gens bien éduqués le font ! »\\
5. « On sort jamais de Céline comme on en (sic) est entré – ça lui plairait beaucoup, ce que je dis là ! »\\
6. « Un hybride c'est comme un rhinocéros : c'est une licorne croisée avec un dragon. Ni l'un, ni l'autre n'existent et pourtant on a un rhino ! Et celui-là, il vaut mieux ne pas le croiser la nuit... »\\
7. *A un élève la main dans le plâtre* « Vous êtes gaucher ? Vous y êtes allé trop fort ? »\\
\end{minipage}\hspace*{.06\textwidth}\begin{minipage}{.14\textwidth}
a. Un Élève\\[\retourLigne]
b. M. Chimiste\\[\retourLigne]
c. M. Kernen\\[\retourLigne]
d. Mme Science\\[\retourLigne]
e. M. Faurax\\[\retourLigne]
f. M. Rütsche\\[\retourLigne]
g. M. Français\\[\retourLigne]
\end{minipage}
}

\newpage


\vspace*{-1.2cm}
\begin{article}
{L'invasion de l'anglicisme dans la langue française}
{Antoine}{chez Bob}
\exerguedebutarticle[Français]{Vous qui vous exprimez dans la langue de Molière, n'avez-vous jamais fait attention à la provenance des mots que vous utilisez ?}

\begin{spacing}{1.05}
\vspace*{1mm}
\icsc{L}{orsqu'on} se balade dans les couloirs, on entend toujours une multitude de discussions simultanées.  Mais la plupart d'entre nous ont ce vilain défaut (s'il peut être qualifié de tel) de remplacer un mot francophone par son homologue étranger, que ce soit en anglais ou dans une autre langue (bien que l'anglais prime de par sa nature internationale) -- Jeunesse bilingue oblige. Voici pour vous quelques conseils tout droit sortis du Parchemin pour remédier à cela et briller lors de vos prochaines dissertations : Ne dites plus checker un mail, dites vérifier ! Remplacez challenge par défi, fun par amusant, week-end par fin de semaine, et utilisez un dictionnaire de langue si l'inspiration vous manque.

Certes, il est vrai que c'est parfois un mal nécessaire d'utiliser le mot d'origine. Par exemple, \enquote{tweet} devrait être traduit par \enquote{gazouillis} (traduction québécoise). Mais puisqu'on en parle, autant connaître des mots francophones que personne ne connaît ! Allez placer poliorcétique, palsambleu, impavide, hérésie, grégaire, dépravation, lubie, calcifier, et savourez le regard troublé, parfois choqué de vos camarades ne lisant pas le Parchemin. Les définitions sont à trouver, elles sont dissimulées dans l’ouvrage qu'est Robert (Bob, pour les intimes).\end{spacing}

\end{article}

\ligne


\vspace*{-5mm}
\begin{article}
{Exoconférence,}
{Gaëtan}{théâtre du Léman}
\exerguedebutarticle[Spectacle]{Alexandre Astier, de passage à Genève, nous présentait vendredi soir son \enquote{Exoconférence}.}

\vspace*{-2mm}
\begin{spacing}{1.05}
\icsc{L}{es portes} s’ouvrent. Salle enfumée et une dernière répétition sûrement toute proche. Tout le monde se dirige vers son siège, celui depuis lequel il aura l’occasion  d’admirer le talent du très attendu comédien. Les lumières s’éteignent, le silence se fait.

Une petite musique et c’est parti. Alexandre Astier entre sur scène, énergique, souriant, un paquet de popcorn à la main.  Le sujet de sa conférence : La vie extra-terrestre.

Une heure et demie durant, il nous fait découvrir l’Univers vu par Alexandre Astier : de Copernic à Minkowski en passant par Pascal. Quelle est l’extraordinaire histoire de sa découverte et quelles sont les questions qu’elle soulève ?

%\exerguemilieuarticle{« Je ne sais qui m'a mis au monde, ni ce que c'est le monde, ni que moi-même; je suis dans une ignorance terrible de toutes choses. »\\-- B. Pascal}

Avec son talent de vulgarisation et de dérision, il peint le portrait de cette humanité pleine de convictions parfois destructrices, mais toujours balayées par la curiosité et l’ingéniosité; cette humanité tantôt grand prodige, tantôt petite, ridicule et effrayée par \enquote{le silence éternel des espaces infinis}.

Avec cet humour qu’il avait prêté au Roi Arthur (de \textit{Kaamelott}) et à Johan Sebastian Bach (dans \textit{Que ma joie demeure}), Alexandre Astier nous a fait voyager. Et depuis le fauteuil où nous étions assis depuis une heure et demie, déjà, il nous annonce que, récemment, du formiate d’éthyle a été découvert dans notre galaxie. La Voie Lactée aurait donc un goût de framboise. Une barquette déposée sur le devant de la scène pour les gourmands du premier rang, un dernier salut au public et le retour toujours souriant et énergique de l’artiste en coulisses sous des applaudissements mérités.

Alexandre Astier est de retour à Genève en septembre. Et en attendant une réponse définitive à la question de la Vie, il nous invite à en profiter à l’aide de ces quelques fruits à la saveur extra-terrestre.\end{spacing}
\end{article}


\vspace*{-2mm}
\encadrecolonnes{3}{L'équipe}{

\textbf{Rédac' cheffe\,:}\\
Alicia Frésard

\textbf{Rédaction\,:} Kita Enrire, Samuel Wanja, Lino Mercolli, Matteo Chancerel, Stella Chapou, Tania DeAlmeida, Charlotte Blessemaille, Mathilde Genoud, Matthis Pasche, Aline Monnard, Hervé Zumbach, Xavier Von der Weid, Renat Arjantsev,  Garance Sallin, Damien Geissbühler, Antoine Morisod, Oriane Rutsche, Yitao Li, Gaëtan Ducrest, Theo Studer

\textbf{Illustration\,:} Audray Stadler, Gaëlle Hoslettler, Thomas Buswell, Annie Sulzer

\textbf{Correction\,:} Aline Monnard, Mathilde Genoud et Garance Sallin

\textbf{Mise en page\,:} Yves Zumbach
}


\newpage
\thispagestyle{noNumber}
\newgeometry{ignoreheadfoot,
	vcentering,
	hmargin=2cm,
	vmargin=3cm,
	bottom=2cm,
	footskip=5mm
	}
\vspace*{-1.2cm}
\rubrique{Mots-croisés}

\vspace*{-3mm}
\begin{minipage}{.48\textwidth}
\tikzstyle {gridstyle}=[thick]
\begin{tikzpicture}[scale=1.32, every node/.style={scale=1.32}]
\hspace{-4mm}
\begin{crossgrid}[h=11,v=11]
\blackcases{2/3, 2/9, 2/11, 3/1, 3/8, 4/4, 4/8, 4/10, 5/4, 5/6, 5/11, 6/1, 6/8, 7/2, 7/8, 8/4, 8/6, 8/7, 9/2, 9/3, 9/10, 10/7, 10/8, 10/9, 11/2}
\tikzstyle{wordstyle}=[color=gray, font=\tiny, anchor=south east, yshift=-3, xshift=1.5]
\words[v]{1/1/g, 2/5/b, 3/4/i, 3/9/k, 4/6/h, 4/11/a, 5/3/e, 6/6/j, 6/10/d, 7/7/f, 9/1/m, 9/5/l, 11/7/c}
\end{crossgrid}
\end{tikzpicture}
\end{minipage}\hspace{.04\textwidth}\begin{minipage}{.48\textwidth}
\textbf{Horizontal}
\vspace*{-3mm}\begin{multicols*}{2}
\maDef{h}{Suisse ; chut en anglais ; fille du cousin germain}\\
\maDef{h}{Survient quand on a trop bu ou trop mangé}\\
\maDef{h}{Assemblées ; prénom des rappeurs}\\
\maDef{h}{Etats-Unis ; dieu du soleil ou cri d’exaspération; comme}\\
\maDef{h}{Découvrir quelque chose par hasard}\\
\maDef{h}{Côté obscur de la force ; organisation pour la sécurité créée en 1917 ; société de l’école public}\\
\maDef{h}{Se trouve dans le corps humain ou la lessive}\\
\maDef{h}{Toi ; unique}\\
\maDef{h}{Animal du parchemin}\\
\maDef{h}{A vécu ; huer au passé simple ; une mole de cet élément pèse 197g}\\
\maDef{h}{Exajoule ; pays limitrophe de la Suisse}
\end{multicols*}
\end{minipage}


\vspace*{-1mm}
\textbf{Vertical}\\
\newlist
\vspace*{-.9cm}\begin{multicols}{4}
\maDef{v}{Petites chausses}\\
\maDef{v}{Exclamation ; possessif allemand}\\
\maDef{v}{Se trouve dans les montres ; acteur maître en art martiaux}\\
\maDef{v}{L’ONU pour le collège ; école d’horticulture pour yack}\\
\maDef{v}{Bonjour en suédois ; vitesse du son}\\
\maDef{v}{Plier ; affirmatif}\\
\maDef{v}{Travaillent dans les mines ou les jardins ; vit dans les villes ou les champs}\\
\maDef{v}{Station spatiale internationale ; syndicat des travailleurs suisses}\\
\maDef{v}{Thé tranquillisant}\\
\maDef{v}{Astre chevelu ; exclamation en argot anglais}\\
\maDef{v}{Montre à eau}\\[2mm]
\textit{Mots-croisés par Dali}
\end{multicols}

\vspace*{-.5cm}
\begin{minipage}{.16\textwidth}
\textbf{Phrase mystère:}
\vspace*{4mm}
\end{minipage}\begin{minipage}{.8\textwidth}
	\begin{Puzzle}{15}{1}
	|[a]q|[b]q|* |[c]q|[d]q|[e]q|[f]q|* |[g]q|[h]q|[i]q|[j]q|[k]q|[l]q|[m]q|.
	\end{Puzzle}
\end{minipage}

\vspace*{-3mm}
\rubrique{Mots-mêlés}

\begin{minipage}{.61\textwidth}
	\renewcommand{\arraystretch}{1.5}
	\setlength{\tabcolsep}{4pt}
	\scriptsize{
	\begin{tabular}{|c*{20}{|c}}
	\hline
	W & D & R & Y & L & E & S & C & A & L & I & E & R & J & A & B & X & N & Z\\\hline
	Z & X & J & K & A & C & E & C & J & F & P & A & R & C & H & E & M & I & N\\\hline
	C & M & S & C & U & A & M & T & F & O & L & Y & W & C & A & C & L & C & D\\\hline
	S & A & E & A & L & R & E & G & H & U & I & U & C & A & X & R & K & O & E\\\hline
	A & T & C & S & A & A & S & Y & W & R & T & Q & Y & F & P & O & C & L & S\\\hline
	L & U & R & I & D & C & T & N & K & C & R & B & W & E & J & I & R & L & T\\\hline
	L & R & E & E & E & O & R & P & P & H & A & I & M & T & R & S & E & E & A\\\hline
	E & I & T & R & S & L & I & A & R & E & V & P & A & E & O & S & D & G & E\\\hline
	Q & T & A & G & C & L & E & I & O & T & A & F & C & R & B & A & A & I & L\\\hline
	U & E & R & L & A & E & L & N & F & T & I & M & A & I & J & N & C & E & R\\\hline
	I & S & I & I & L & G & L & A & E & E & L & E & R & A & E & T & T & N & E\\\hline
	P & T & A & C & A & I & E & U & S & V & D & D & O & L & T & U & R & S & N\\\hline
	U & T & T & O & D & E & S & C & S & E & E & I & N & V & S & Q & I & I & N\\\hline
	E & K & K & R & E & N & A & H & E & R & M & A & C & X & T & D & C & I & U\\\hline
	P & H & I & N & I & N & G & O & U & T & A & T & H & C & R & G & E & X & I\\\hline
	P & K & A & E & S & E & X & C & R & E & T & H & A & C & O & Q & P & F & F\\\hline
	S & B & K & L & G & S & A & O & S & Ç & U & E & R & T & U & M & N & H & B\\\hline
	B & N & P & L & L & G & P & L & B & F & Q & Q & L & K & V & L & P & U & F\\\hline
	X & I & G & P & L & A & G & A & E & W & R & U & I & K & E & I & B & T & W\\\hline
	T & N & M & G & H & E & K & T & O & V & B & E & E & F & S & D & M & W & A\\\hline
	\end{tabular}
	}
\end{minipage}\begin{minipage}{.41\textwidth}
	\begin{multicols*}{2}
		\textsc{aula
		cafeteria\\
		carac\\
		casier\\
		charlie\\
		collegiennes\\
		collegiens\\
		compta\\
		croissant\\
		destael\\
		ennui\\
		escalade\\
		escalier\\
		fourchette verte\\
		hall\\
		licorne\\
		macaron\\
		maturites\\
		mediatheque\\
		objets trouves\\
		pain au chocolat\\
		parchemin\columnbreak\\
		professeurs\\
		redactrice\\
		salle qui pue\\
		secretariat\\
		semestrielles\\
		ski\\
		travail de matu}
	\end{multicols*}
\end{minipage}


\end{document}