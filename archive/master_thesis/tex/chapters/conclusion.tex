\nomargintoc
\nomarginimagecaption%
\chapter{Conclusion}
\label{sec:conclusion}

\leavevmode\marginElement{Open source represents critical infrastructure for today's societies, yet we fail to protect it accordingly.}%
\marginElement{\input{images/sectioning/horizontal/\arabic{image_sectioning_horizontal}/readme.tex}}%
It is a rather undisputed fact that the world runs on open source.
To give only one example, it is estimated that WordPress, an open source website framework, ran approximately 42\% of all the websites in October 2021 \cite{noauthor_wordpress_2022}.
Any software today, somewhere in its dependencies, uses a piece of open source code, if only because the very majority of the most used programming languages are open source.
Nevertheless, as a society, we do not yet act accordingly; we fail to acknowledge how critical the infrastructure that powers our societies is, and so we do not protect it enough.
\textcquote{filippo_valsorda_professional_2021}{The status quo is unmaintainable.}
We need a mentality shift, we need to inject more money into the open source, and we need to build incentive systems that provide us with more guarantees.

\leavevmode\marginElement{The goal of this work: improve open source \emph{trustlessness} by designing new, blockchain-based coordination systems.}%
This is what we call \emph{making open source trustless}: being able to trust open source to create only positive outputs, without having to trust any individual contributors.
If open source becomes trustless, then society is better off.
To improve the trustlessness of open source, we proposed in this work to focus on designing new coordination systems.
We used blockchain as the technological foundation because it is trustless itself.
Using any trustful technology instead would prevent us from achieving trustlessness.

\leavevmode\marginElement{Coordination systems are hard.}%
Unfortunately, designing good coordination systems is difficult.
This research field needs to combine math, game theory, psychology, philosophy, economics, and computer science.
There are a great variety of properties that we wish these systems to offer: security in the face of adversarial agents, Pareto efficiency, minimum friction, etc.
And we know that we cannot have it all: nature imposes some fundamental limitations, like Arrow's impossibility theorem; some properties conflict with one another, like privacy and transparency; and we must design those systems around human limitations, like the recency bias.

\leavevmode\marginElement{Decentralization is a key to trustlessness.}%
Bringing \emph{trustlessness} to open source, our ultimate goal, requires at least \emph{decentralization}; indeed, a system cannot be trustless if it is centralized.
We proposed that there are two key components to being decentralized.
First, there need to be many actors involved; second, each actor needs to have the same power as the others.
We also put forward, that building systems that naturally revert to a decentralized state, i.e.\ what we defined as \emph{progressive decentralization}, is an approach more resilient, than building a system that is decentralized to begin with, and hoping that it remains so afterward.
Especially knowing that power often acts as a reinforcement loop, i.e.\ having power enables getting more power.

\leavevmode\marginElement{Decentralization and open source.}%
Open source projects always begin in a centralized state anyways, because they are initiated by at most a few individuals.
So building initially decentralized systems is hopeless, and only progressive decentralization can bring decentralization about.
Through the core/halo categorization, it is understood that most open source projects today never managed to decentralize: the power is in the hands of a few core developers.
There is much that can be improved.

\leavevmode\marginElement{Importance of \textit{proof of personhood}.}%
In this work, we proposed that an intrinsic progressive decentralization system, so a system in which the power distribution reverts to the uniform distribution unless power differences are continuously maintained over time, through repeated, valuable contributions, can only be built, if a satisfying answer to \textit{proof of personhood} is found.

\leavevmode\marginElement{Proposed blockchain primitives.}%
While intrinsic progressive decentralization is currently not achievable, this work proposed primitives that improve various aspects of trustlessness:

\begin{description}
  \item[Decreasing Power Governance Tokens]
    provide incentives for newcomers to participate, and for members to contribute regularly.
    This improves the first aspect of decentralization: widening the community of a project and shrinking the gap between highly active and less active members, by incentivizing regular contributions.
    By distributing the power to all those that participate, the system is an improvement over the Benevolent Dicta\-tor for Life model that most open source projects use today.
  \item[The Voting Workflow]
    specifically suited to merge requests improves the security guarantees over the code accepted.
    The challenge mechanism further deters adversarial behaviors.
  \item[The Rewarding Scheme]
    to award tokens in compensation for the value provided to a project realigns value creation and value extraction.
    This creates incentives to provide more value.
  \item[The Money Distribution Process]
    rewards developers in a fair way.
    Paying developers can transform contributing to open source into a sustainable activity, thus it improves projects' longevity.
    We also discussed the possibility of using Radicle Drips to split part of the received money and donate it to the dependencies of a project.
    This creates a funding graph through the open source ecosystem, thus even the most hidden, but fundamental, open source libraries could receive money.
  \item[Issue Backing]
    provides a way for users to communicate preferences to developers, which makes it possible for open source projects to provide more value to society.
    It also increases the amount of money sent to open source projects and their developers.
\end{description}

\leavevmode\marginElement{Solving cooperation issues at large.}%
The 21\textsuperscript{st} century will be one of many challenges.
The human population is larger than ever, and problems scale accordingly.
Technological developments have made us more powerful than ever and the consequences of our actions scale accordingly.
This is why coordination failures today have more dramatic consequences: wars are more destructive, economical crises impact more humans.
Climate change is the embodiment of the impacts that failing to coordinate around a public good yield.
We need to improve our governance systems if we are to solve these issues.
We need governments and companies that are more representative, more transparent, more efficient, more aligned with humanity's shared goals, and more cooperative among themselves.
We hope that the primitives listed above contribute to making open source more \emph{trustless}.
We hope that they improve human coordination.
